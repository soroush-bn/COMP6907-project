\documentclass{article}
\usepackage{amsmath}

\begin{document}

This section of the paper focuses on the development and evaluation of a neuromusculoskeletal (NMSK) estimator for individual muscle force estimation in forearm muscles during isometric contractions. The objective was to address the challenge of accurately estimating muscle forces when measurements are only available from a subset of the muscles involved. The NMSK estimator integrates forward dynamics estimation with a neural model of muscle cocontraction to improve accuracy even with incomplete muscle measurements. A computational framework was developed to assess the impact of physiological variability, cross-talk, and measurement error on the estimator's performance through sensitivity analysis. Results showed that the NMSK estimator reduced estimation error by 25\% on average in noise conditions and was robust against physiological variability, outperforming a standard estimator. This research has implications for enhancing muscle force estimation accuracy in future applications involving coordinated movements of forearm muscles. In terms of mathematics, the estimation error reduction of 25\% can be represented as:

\[ \text{Error Reduction Percentage} = \left( \frac{E_{\text{std}} - E_{\text{NMSK}}}{E_{\text{std}}} \right) \times 100\% \]

where \(E_{\text{std}}\) represents the error of the standard estimator and \(E_{\text{NMSK}}\) represents the error of the NMSK estimator.

\end{document}