\documentclass{article}
\usepackage{amsmath}
\usepackage{amsfonts}
\usepackage{amssymb}
\usepackage{graphicx}

\begin{document}

The detection of radio emissions from exoplanets around M dwarfs, particularly those with rocky compositions, remains elusive despite decades of efforts. However, the discovery of auroral radio emissions from these exoplanets would confirm their existence and provide insight into their intrinsic magnetic fields, a piece of information not obtainable through other means. Unfortunately, Earth and super-Earth exoplanets with magnetic fields as strong as those of Earth would emit radio waves below 10 MHz, making them undetectable from Earth. However, if an exoplanet is close enough to its host star to be in the sub-Alfvénic regime, where the plasma speed relative to the exoplanet is less than the Alfvén speed at the planet's position, energy and momentum can be transported upstream by Alfvén waves, resulting in auroral radio emissions similar to Jupiter's interactions with its Galilean satellites.

M dwarfs are ideal candidates for detecting star-planet interactions at radio wavelengths due to their strong surface magnetic fields, which can generate cyclotron emission with frequencies that fall within the range of current radio interferometers. Sub-Alfvénic interactions, which involve Alfvén waves connecting the star to the planet, can produce detectable periodic radio emission using the ECM mechanism. This emission could be confirmed through strong, highly circularly polarized radio signals at frequencies close to or below the maximum cyclotron frequency, with periodic signals correlating with the planet's orbital period or synodic period, indicating the presence of a magnetic field.

\end{document}