\documentclass{article}
\usepackage{amsmath}
\usepackage{amssymb}
\begin{document}

The discovery of exoplanets around M dwarfs, including rocky ones, has primarily relied on radial velocity and transit measurements, while the detection of auroral radio emission from these planets using radio observations remains elusive despite decades of efforts. The detection of auroral radio emission from an exoplanet would confirm its existence and provide information about its intrinsic magnetic field, which is not obtainable through other means. However, Earth and super-Earth exoplanets with plausible magnetic field strengths up to a few Gauss would emit radio frequency auroral emissions below the Earth's ionospheric cutoff frequency of 10 MHz, making this radio emission undetectable from Earth. If an exoplanet is close enough to its host star, such that it is in the sub-Alfvénic regime, where the plasma speed relative to the exoplanet is less than the Alfvén speed at the planet's position, energy and momentum can be transported upstream back to the star by Alfvén waves. This sub-Alfvénic interaction, as observed in Jupiter's interaction with its Galilean satellites, results in copious auroral radio emission. In summary, M dwarfs, due to their strong surface magnetic fields, offer a promising platform for detecting star-planet interactions via cyclotron emission at radio wavelengths. This phenomenon is expected to produce periodic radio signals that can be detected by existing radio interferometers, potentially providing a novel method for exoplanet detection. The confirmation of such signals would involve observing highly circularly polarized radio signals, and this could be achieved through future radio surveys of M dwarf systems.

\end{document}